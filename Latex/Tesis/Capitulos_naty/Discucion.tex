% !TEX root = ../Tesis_NataliaOpazo.tex


La depositación de polvo aéreo es una fuente importante de hierro para el oc\'eano, particularmente en las zonas HNLC \citep{jin2008impact,martinez2014iron,lambert2015dust}. \'Esto dato que ejerce un control en la biología marina por su baja concentración, convirtiéndose en un elemento limitante \citep{falkowski1998biogeochemical,martin1990glacial,gruber2008marine,tagliabue2017integral}.

El incremento de flujos de polvo durante periodos glaciares \citep{mahowald1999dust,gaspari2006atmospheric,lambert2008dust,maher2010global,lamy2014increased} pudo haber estimulado la productividad primaria y/o la producción de exportación de materia orgánica, por lo que puede ser un factor determinante en la diferencia entre 80 y 100 ppm \citep{sigman2000glacial,hain2010carbon,ferrari2014antarctic} de concentración atmosférica de CO$_2$ durante periodo glaciares e interglaciares.

Con el propósito de evaluar la sensibilidad de la bomba biológica al suministro de hierro y cómo este afecta la concentración de pCO$_2$ atmosférico, es que realizó una prueba mediante el modelo biogeoquímico de complejidad intermedia cGENIE durante el periodo que abarca desde el UMG y el comienzo del Holoceno, a partir, de cinco campos de flujos de polvo \citep{lambert2015dust,yukimoto2012new,sueyoshi2013set,takemura2009simulation,albani2014improved} provenientes tanto de modelos como de recontrucciones. Obteniéndose un promedio máximo de reducción de pCO$_2$ a nivel global en torno a los 17 ppm. Valor que está ligeramente por sobre lo estimado por otros trabajos desarrollado por medio de modelos GCM más complejos como PISCES y los desarrollados por MIT, a partir de los cuales se estima una captura por parte de los océanos correspondiente a aproximadamente 8 \citep{parekh2006atmospheric,lambert2015dust}, 11 \citep{tagliabue2009quantifying} y 15 ppm \citep{bopp2003dust}. Por otro lado, modelos de caja como \cite{hain2010carbon} estima una captura de alrededor de 35 ppm, valor que está muy por sobre los máximos 21 ppm evaluados por medio de los flujos de polvo desarrollados por \cite{albani2014improved}. 

Estos resultados estarían reflejando que la mayor liberación de polvo sobres los océanos superficiales durante el LGM habría permitido mejorar la utilización de NO$_{3}^{-}$ y PO$_{4}^{3-}$, debido a un aumento en la disponibilidad de hierro. Macro y micronutrientes que en conjunto con el CO$_2$ disuelto forman parte del fitoplancton lo que permitiría dicha disminución en la concentración atmosférica de CO$_2$.

Cálculos de los aportes regionales en la captura de pCO$_2$ muestran que son las altas latitudes (Océanos del Sur y Pacífico Norte) las que mayor control ejercen sobre la variabilidad de este gas, lo que se corresponde con lo mostrado por \cite{lambert2015dust} y \cite{bopp2003dust}. 

Si bien es sabido que en regiones subpolares como los Océanos del Sur, aguas profundas afloran llevando grandes contenidos de nutrientes a las superficies en estas regiones. Proceso que podría producir una liberación de CO$_2$ hacia la atmósfera, además de un gran suministro de nutrientes que en la actualidad es rápidamente devuelto a las profundidades del océanos sin lograr ser utilizados para la formación de biomasa \citep{hain2010carbon}. La mayor captura durante el UMG podría deberse, por un lado, a que hubo un aumento en la cobertura de hielo en la zona Antártica \citep{ferrari2014antarctic}, lo que habría generado un bloqueo y eventual desplazamiento de gran parte de la surgencia de agua profunda hacia el océano subtropical, conllevando a un menor suministro de CO$_2$ (evitando una liberación de éste desde las aguas oceánicas superficiales hacia la atmósfera), y por otro lado, un menor suministro de nutrientes \citep{tagliabue2009quantifying} que habría sido mejor utilizado debido a la mayor tasa de polvo durante este periodo, que habría mejorado la fertilización con hierro a los océanos provocando un aumento del flujo de PE \citep{martin1990glacial,toggweiler2006midlatitude,shaffer2018and}. 

 Además en las zonas HNLC limitadas por hierro existe una predominancia de diatomeas \citep{bopp2003dust,arellano2011high} para las cuales la concentración de Si(OH)$_4$ es limitante. Por esta razón, se ha propuesto que durante el UMG la mayor utilización de NO$_{3}^{-}$ y PO$_{4}^{3-}$ por otros organismo planctónicos podría haber dejado un excedente de Si(OH)$_4$ que producto de la circulación marina podría haber sido transportado a zonas con limitación de Si(OH)$_4$ como las regiones subtropicales de giros oligotróficos, provocando una captura general mayor de pCO$_2$ atmosférico entre el UMG y el Holoceno \citep{matsumoto2002silicic}, sin embargo, hay estudios que se contraponen a esta idea como \cite{tagliabue2014impact} que muestra que una mayor deposición de hierro en altas latitudes, provoca una disminución de Si producto de una mayor utilización, generando un menor transporte a bajas latitudes. 

Los resultados presentados, son valores que están sujetos a la variabilidad inducida por los propios modelos de polvo utilizados como forzantes, en este caso mediante cGENIE, los cuales sobrestiman en la mayoría de los casos los niveles de depositación de polvo en latitudes del norte y subestiman las fuentes glaciogénicas del hemisferio sur durante el LGM, razón por la cual este trabajo es un primer acercamiento en torno al efecto del hierro proveniente de fuentes eólicas en la bomba biológica. No obstante, tanto la concentración de ligandos como la fuente de Fe son factores que pueden cambiar y que tienen una consecuencia en la PE. Finalmente, si bien vemos que la bomba biológica tiene un impacto en la reducción de CO$_2$ este efecto no alcanza a explicar toda la diferencia entre el UMG y el Holoceno, por lo tanto, otros factores como la alcalinidad del océano, la bomba de carbonato y/o mecanismo físicos como la estratificaci\'on pueden estar actuando para explicar esta variabilidad.  

Las dos hipótesis planteadas en este trabajo son aceptadas (H$_{0}$), dado que ambas se cumplen. Como vimos, existe un efecto del polvo en el $\Delta$pCO$_{2}$ del periodo que abarca entre el UMG y el Holoceno. El $\Delta$pCO$_{2}$ generado proviene en su mayoría de los cambios en los océanos del sur. 



