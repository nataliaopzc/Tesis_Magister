% !TEX root = ../Tesis_NataliaOpazo.tex

El dióxido de carbono (CO$_2$) es el gas m\'as importante de efecto invernadero de acuerdo con el \textit{Panel Intergubernamental del Cambio climático} \citep{IPCC2014}. 
Este gas, de origen principalmente antropogénico ha ido incrementando desde tiempos preindustriales (1750) \citep{luthi2008high}. Lo que generó las primeras alarmas en torno a la crisis ambiental, impulsadas por los informes de la Conferencia sobre el Medio Humano de la ONU, realizada en Estocolmo 1972. Estos informes mostraron la gravedad, situación general y proyecciones del clima. Luego con el informe de Brundtland (1987), donde se contrastaron posturas de desarrollo económico con sustentabilidad ambiental \citep{Pierri2005}, se fue cimentando el concepto de cambio climático, lo que en la actualidad se ha masificado e impulsado un sin números de estudios, proyecciones y visiones en torno a la problemática. En octubre del 2017 el boletín de la Organización Meteorológica Mundial publicó que la concentración atmosférica de CO$_{2}$ alcanzó un promedio anual de 403.3 ppm (WMO, 2017). Consiguiendo en la actualidad exceder la presi\'on parcial de CO$_2$ (pCO$_2$) de los últimos 800000 años, un periodo que ha sido caracterizado por una gran y cíclica variación climática que ha oscilado entre 180 y 280 ppmv aproximadamente, que corresponde a periodos glaciares e interglaciares respectivamente \citep{harrison2000role,ferrari2014antarctic}. 

A lo largo de la historia geológica han existido mecanismos naturales que han actuado para contener el CO$_2$ atmosférico. El océano ha sido el principal reservorio, conteniendo aproximadamente cincuenta veces más carbón que la atmósfera y casi veinte veces más que la biósfera terrestre \citep{broecker1980modeling}. Lo anterior es, en parte, consecuencia del balance natural de pCO$_2$ que tiende a existir en medio de la superficie del océano y la capa atmosférica suprayacente. Cuando acontece una transferencia de carbón desde la atmósfera hacia el océano, uno de los medios por el cual este carbón es removido de la capa superficial oceánica para, eventualmente, llegar al fondo oceánico es la formación de materia orgánica a partir de la fijación fotosintética del CO$_2$ (bomba de tejidos blandos) \citep{falkowski1998biogeochemical,anderson2002southern,sigman2003biological,kohfeld2005role}. 

Existen áreas en el océano que están limitadas por macro y micronutrientes esenciales para la eficiencia de la productividad primaria. Tal es el caso del hierro \citep{archer2000caused,jickells2005global,martinez2014iron}, micronutriente cuya importancia en la bomba de tejidos blandos est\'a estrechamente vinculada a la formaci\'on de derivados del nitr\'ogeno y ligandos en el oc\'eano. Su origen son, principalmente, las fuentes hidrotermales, los m\'argenes continentales y los flujos de polvo eólico. Este \'ultimo es importante como suministro superficial de hierro al oc\'eano abierto \citep{mahowald2011aerosol,prospero2002environmental,tagliabue2017integral}. A partir de mediciones obtenidas de hielo derretido de la Ant\'artica \citep{augustin2004eight}, \cite{lambert2008dust} mostraron que altas y bajas concentraciones de polvo reflejan ciclos glaciares e interglaciares respectivamente. De lo anterior, se cree que existe una relaci\'on entre el hierro y gran parte de la diferencia de pCO$_2$ entre 80 y 100 ppmv \citep{shaffer2018and}, ya que habría aliviado la limitación de hierro de vastas \'areas del océano y en particular de las zonas con \textit{alto contenido de nutrientes y baja concentración de clorofila} (HNLC, por sus siglas en inglés) durante el UMG \citep{martin1990glacial}.

Desde mediados de los años 90s, se comenzaron a simular las variaciones atmosféricas de CO$_2$ en el clima mediante modelos numéricos. Sin embargo, aún en la actualidad, las causas de la variabilidad del CO$_2$ entre tiempos glaciares e interglaciares permanecen sin ser del todo comprendidas y, por lo tanto, se ha incorporado la biogeoquímica como un mecanismo adicional a los procesos f\'isicos de la din\'amica oc\'eano-atm\'osfera. En este sentido, se han desarrollado diferentes tipos de modelos num\'ericos, tanto de alta como de baja complejidad, con el prop\'osito de comprender y predecir las posibles variaciones clim\'aticas producto de la adici\'on de alg\'un proceso que puede forzar el clima. Entre \'estos est\'an los \textit{General Circulation models} (GCM) acoplados, los cuales tienen un costo computacional alto y grandes incertidumbres. Sin embargo, éstos han demostrado tener un potencial importante en t\'erminos de incorporar retroalimentaciones del clima y/o ciclo del carbono. Por otro lado, están los modelos de cajas, los cuales dividen el sistema Tierra en un conjunto de subsistemas (cajas) que son altamente eficientes computacionalmente, pudiendo operar en un orden de 10000 o más años, pero tienen fuertes limitaciones dado que tienden a homogeneizar los volúmenes oceánicos \citep{weber2010utility}. Los \textit{Earth System Model of intermediate complexity} (EMIC) tienen una reducida parametrización y operan en escalas de tiempo de hasta aproximadamente 10000 años, sin embargo son relativamente costosos computacionalmente en modelamientos de circulación oceánica en 3-D.  

En este estudio, cuantificamos el efecto de la inclusión de flujos de polvo globales en la superficie del océano sobre el CO$_2$ atmosférico. Para ello, se usó el modelo cGENIE, un EMIC con énfasis en el ciclo del carbono \citep{ridgwell2007marine}. En este modelo se utilizaron campos de flujos de polvo globales del periodo que abarca desde el fin del Pleistoceno (UMG, aproximadamente 21000 años atrás) hasta el periodo pre-industrial \citep{sigman2000glacial,lynch2007atlantic,braconnot2007results,barker2009interhemispheric}. Estos datos fueron obtenidos, tanto de simulaciones de polvo como de reconstrucciones basadas en datos observacionales. El fin de lo anterior fue comparar las incertidumbres asociadas a las estimaciones de CO$_2$ mediante diferentes métodos.
$\Delta$pCO$_{2}$
\section{Hipótesis}

Bajo la hip\'otesis desarrollada por \cite{martin1990glacial}, que describe el efecto que tiene el hierro como micronutriente en los tejidos orgánicos de la biología oceánica.
En este trabajo de investigación, se trabajará con las dos hipótesis propuestas a continuación: 
\begin{itemize}
	\item[$H_{o}$] Existe un efecto del polvo en el $\Delta$pCO$_{2}$. \\
El polvo es uno de los medios por el cual se suministra hierro en la superficie del océano. Éste micronutriente se considera limitante, dado que en algunas regiones oceánicas condiciona la productividad biológica.  No obstante, el carbono es, a su vez, parte de las razones de \textit{Redfield}, por lo tanto, tiene una directa relación con la productividad de materia orgánica. De esta manera, tanto el hierro como el carbono se encontrarían relacionados, y se espera conocer su grado de interacción. 
\item[$H_{o}$] El $\Delta$pCO$_{2}$ generado entre el UMG y Holoceno debido al efecto del polvo, proviene de los cambios en los océanos del sur. \\
Existen áreas del océanos que tiene muchos nutrientes no utilizados disponibles para la formación de materia orgánica. Estas regiones, son conocidas como zonas con alto contenido de nutrientes, bajas concentraciones de clorofila. Entre éstas, destacan por su extensión los océanos del sur. Es por esta razón, que se espera tengan un impacto mayor en la variabilidad del pCO$_2$.  
	\end{itemize} 

\section{Objetivos}

\subsection{Objetivo general}

Determinar el rol que las fuentes de polvo han ejercido en los balances biogeoquímicos de la bomba de tejidos blandos del océano, en la diferencia entre 80 - 100 ppm de concentración atmosféroca de CO$_2$ durante el UMG hasta Holoceno. 


\subsection{Objetivos espec\'ificos}

\begin{itemize}
  \item{\bf I.} Cuantificar mediante un EMIC, con la adición del ciclo del carbono, el nivel de captura de CO$_2$ de los océanos producto de la bomba de tejidos blandos. 
  \item{\bf II.} Calcular la contribución de cada región HNLC a la diferencia de CO$_2$ debido a los distintos flujos de polvo. Se utiliza para lo anterior, datos de polvo provenientes de observaciones y de simulaciones.
 \item{\bf III} Determinar la diferencia de CO$_2$ existente entre el UMG y el Holoceno. 
  \end{itemize}


